\chapter{Experiments}

All our reality gap approaches have been validated on a robotic application with the Thymio robot on an obstacle avoidance task. The experimental set up among all approaches mainly differ in the evolutionary strategy used or whether the optimization happened fully or partially in simulation and reality. The following next paragraphs describe the common set-up used in our experiments.

The robot is quipted with 7 infrared sensors and 2 two motors. The virtual thymio robot is modeled based on the physical thymio and its characteristics \ref{thymio_characteristics}. The specifications of sensor readings and possible speed values, included normalization values are depicted in the following table \ref{fig:thymio_specs} both for the physical thymio and its virtual counterpart.

\begin{table}[H]
\begin{tabular}{llll}
\centering
\hline
\textbf{}                            & \textbf{Simulator}    & \textbf{Reality}  & \textbf{Normalized}  \\ \hline
\textbf{Speed Values}                & {[}-2.0, 2.0{]}       & {[}-200, 200{]}      & {[}0.0, 1.0{]} \\
\textbf{Sensory Readings}            & {[}0.0, 1.0{]}        & {[}0, 4500{]}        & {[}0.0, 1.0{]} \\
\end{tabular}
\caption{The Thymio robot sensors and speed values specification.}
\label{fig:thymio_specs}
\end{table}

The evaluation of a single genome takes 60 simulation seconds for the simulation-based approaches and 60 wall clock seconds for the reality-based approaches from a fixed initial position each test. At each step (50ms), the Thymio robot sensory readings are fed to the neural network and the network outputs are multiplied by the maximal wheel speed and used to apply to the wheels of the Thymio robot. The network outputs are in the range of \emph{0.0} and \emph{1.0}, and since the Thymio can move backward, the output values are transformed into a positive/negative range where \emph{0.5} is the point of direction inversion. In order to speed up the optimization, the evaluation of a genome is stopped if the system detects that the robot collided with the walls or the environmental objects. Furthermore, if the robot is not moving or spinning in one place the evaluation is stopped and the fitness is calculated over reduced simulation time.

The fitness function relies on a set of features that can be measured within the interaction between the robot and its environment. Hence the fitness function relies on 3 features, as follows:

\begin{enumerate}
    \item \(V_{t} = \frac{V_{l} + V_{}r}{2} \) \textbf{Average wheel speed} of both wheels at a particular timestamp \emph{t}.
	\item \((1-\sqrt{\Delta v})\) \textbf{The algebraic difference} between the speed values at a particular timestamp \emph{t}. The smaller is the difference, the faster the robot moves.
	\item \((1 - P_{t})\) \textbf{Max Sensor activation} the activation value of the proximity sensor with the highest activity. The closer the robot is to the walls or obstacles the less fitness it accumulates.
\end{enumerate}

The fitness function \ref{fitness_function} is defined as a dot product of all these features divided by the fixed simulation time. 

\begin{equation}
	\[ f = \sum_{t=1}^{t_{max}} \frac{V_{t} (1 - \sqrt{\Delta v}) (1 - P_{t})}{60} \]
	\label{fitness_function}
	\caption{Task dependent fitness function}
\end{equation}

For each evaluated controller, we extract certain \emph{behavioral feautures} \ref{behavioral_features} during the optimization process. These features allow us to describe/quantify the behavior of each controller. Moreover, it allows us to compare controllers in a simple manner without any dependency on the evolutionary strategy applied or the controller's genotype. The 12-dimensional features are defined as 1) the average left and right wheel speeds 2) the average value of each sensor activation 3) percentage of time spent in each section of the arena throughout the evaluation. These features are normalized in the range of \emph{0.0} and \emph{1.0}.

\begin{equation}
	\[ b_{controller} = [avg_{left}, avg_{right}, s_{1}, s_{2}, s_{3}, s_{4}, s_{5}, s_{6}, s_{7}, area_{0}, area_{1}, area_{2}]\]
	\label{behavioral_features}
	\caption{12-dimensional behavarioral features}
\end{equation}

Additionally, we extract the position of the Thymio robot during evaluation. The vision system is used to extract the position from reality and the client is responsible for the position from the simulation. This feature is used to qualify a given controllers transfer from simulation to reality.

\section{Simulation-based optimization}

Our first application aims to explore the \emph{simulation-based optimization} approach to the reality gap problem. The first part of the experiments involves evolving an obstacle avoidance behavior in simulation. Finding the right parameters of the evolutionary strategy to achieve optimal obstacle avoidance behavior, like, number of generations, population number, etc. The second part is focused on the transferability of controllers. Taking the best controllers evolved in simulation and validate how well they transfer to reality. Additionally, this approach is later compared to the reality-based optimization approach to the same application.

\subsection{Experimenal design}

The first experiment takes place fully in simulation using the simulation model mentioned earlier \ref{fig:virtual_arena}. The aforementioned evolutionary algorithm applied in this experiment is \emph{NEAT} \cite{stanley2002evolving}. A controller, in this case, the genome, is represented as a neural network. The implementation is based on the python-neat \footnote{\url{https://neat-python.readthedocs.io/en/latest/}} library which has been modified to fit our requirements. The genome contains a list of \emph{connection genes} and list of \emph{node genes}. Node genes encode the input, hidden nodes, and outputs of the neural network that can be connected. Whether a node is connected to other node is expressed in the connection gene. The initial neural network structure consists of 7 input nodes (each represents the infrared sensors placed around the Thymio), fully connected to the 2 motor neurons computing the speeds of the wheels. This way the algorithm starts with a fixed topology and by applying the biological operators over generations it may evolve. In the early stage of the experiments various parameters of the evolution have been tested (population size, generations, mutation rate, etc.), but the most promising results have been achieved by the parameters summarized in table \ref{neat_parameters}.

\begin{table}[H]
\begin{tabular}{llll}
\centering
\hline
\textbf{}                            & \textbf{Simulator}   & \textbf{Reality}  & \textbf{Normalized}  \\ \hline
\textbf{Speed Values}                & {[}-2.0, 2.0{]}       & {[}-200, 200{]}      & {[}0.0, 1.0{]} \\
\textbf{Sensory Readings}            & {[}0.0, 1.0{]}        & {[}0, 4500{]}        & {[}0.0, 1.0{]} \\
\end{tabular}
\caption{The Thymio robot sensors and speed values specification.}
\label{neat_parameters}
\end{table}

