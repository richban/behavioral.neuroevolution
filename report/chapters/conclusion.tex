\chapter{Conclusion}

In this thesis, we have built an automated robotic platform with features that enhances the ability to test and validate evolutionary optimization procedures both in simulation and in reality, run the entire optimization process on the physical robot without human intervention. Furthermore, to perform a completely autonomous co-evolution of controllers which can take place in simulation and in reality too. This proved to be useful for the robot-in-the-loop simulation-based optimization approaches, especially for the \emph{Transferability approach}. In which we have been able to seamlessly conduct many transfer experiments during the optimization procedure. The mixed reality module of the system enabled to minimize the evaluation time of the overall experiments.

We've spent a significant portion of the thesis into the construction of the automated robotic platform. Combining the various software frameworks to manage the reality-based optimization and co-evolution of the simulation tools. Various tests and modifications have been performed to validate the robustness and functionality of the platform. Ultimately the automated robotic platform is a contribution towards the application of different reality-gap methods for passing the reality gap.

Furthermore, in this thesis, we have addressed the reality gap problem by performing simulation-based, reality-based and robot-in-the-loop optimization procedures on an obstacle avoidance task. We concluded benchmark experiments in simulation and reality using the NEAT algorithm. These benchmarks have been used as a baseline to compare the results with the transferability approach. This approach has been reproduced based on the original implementation \citep{koos2012transferability}. We have discovered that many things can go wrong within this approach, but all in all we were able to confirm our hypothesis. Better results were achieved by the \emph{Transferability} approach when more transfers were conducted on the physical robot. Based on the successful results, we have evolved controllers with no transferability issues. Thus we can come to the conclusion that \emph{transferability} approach is indeed the relevant technique to cross the reality gap with a more accurate surrogate model.

The whole project is hosted on github \footnote{https://github.com/richban/behavioral.neuroevolution}.

