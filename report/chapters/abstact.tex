\cleardoublepage
\chapter*{Abstact}

\thispagestyle{empty}
\vspace{1cm}

In evolutionary robotics,  controllers are evolved using various evolutionary and neuroevolution techniques. Common to these methodologies is the combination with multi-objective optimization algorithms in order to focus on the desired objectives of a given behavior. Such evolutionary optimization procedures can be applied entirely on a physical robot as ER is concerned. However, this procedure is often too time-consuming. An alternative procedure that many research advocates is to speed up the optimization procedure using simulation tools, which has the benefit of being relatively fast and enables to perform a large number of evaluations before an optimal solution is found. While this procedure is a tempting way to mitigate the long procedure of reality-based optimization. It has some serious issues. Among the most well-know one is called \emph{Reality-gap}. Although successful controllers are evolved in simulation, the results are often sealed in simulation, because of defects of the simulation or inaccurate models. The evolved behaviors in simulation do not correspond to reality.

Researches have already addressed the issues on the reality gap, with my thesis I want to take it further and explore/compare the state-of-the-art approaches to the \emph{reality gap} and \emph{transferability} by evolving controllers using neuroevolution. The transferability is concerned with simulation-to-reality disparities of controllers. Our approach aims to find controllers that are both efficient in simulation and transferable from simulation to reality. The focus is on three main approaches \emph{Reality-based, Simulation-based, Robot-in-the-loop} optimization process.  All the approaches are validated on a robotic obstacle avoidance task. While much of the recent works have focused mainly on developing approaches to minimize the transferability gap. A less explored area is in the context of automation/coevolution of the reality-based and simulation-based optimization process. The aim of my thesis is to contribute to the ER field by constructing an automated robotic platform which enables us to test and validate evolutionary optimization procedures both in simulation and in the physical world, run the entire optimization process on the physical robot without human intervention. Furthermore, to perform a completely autonomous co-evolution of controllers which takes place in simulation and in reality too. This could prove to be useful for the robot-in-the-loop simulation-based optimization approaches. A novel element of the thesis is that the reality-based optimization will involve mixed reality system. This will potentially enrich the environment and visualization moreover minimize the evaluation time to reduce the overall experiment. The system consists of several components, such as computer vision, to keep track of the environment and to provide fitness measurements used in fitness calculation, re-configuration and manipulation of the environment. Additionally, to assist the evaluation of controllers and to extract valuable data throughout the experiments. The system has been evaluated through a series of experiments that address the ability to evolve ANN controllers that inhabits an obstacle avoidance behavior.