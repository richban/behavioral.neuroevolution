\chapter{Introduction}

Evolutionary Robotics concerns with the use of Evolutionary Algorithms in robotics. The use of EA in robotics is motivated by a number of issues \cite{meyer1998evolutionary} \cite{grefenstette1994evolutionary}. One of the issues is concerned with the difficulties of hard-coding the control architecture of a robot that has to solve a given task in an unknown or possible changing environment. Because it's impossible to predict each problem that a robot might encounter in a constantly changing environment. More precisely, it's difficult to program a robot control system that can foresee every possible state of its environment and the action it should take.  Another issue is related to simulation models. Building pre-defined abstract models of the world is not sufficient in a continuously changing environment since these models often fail to reflect the complexities of the real world, such as light, gravity, noise, and errors in real sensors, actuators, etc. In response to such difficulties, researchers advocate the use of EA for the development of robots that can adapt and evolve behaviors based on specific environment and problems it's facing.

Evolutionary algorithms are indeed attractive optimization procedures to apply, that merely rewards each evaluated solution with a value that reflects its performance based on a fitness function. However, this procedure often requires a large number of evaluations before optimal solutions are found. As ER concerns robots, it should theoretically be evaluated on physical robot \cite{floreano1998evolutionary}. In practice, the optimization process on a physical robot can be very time-consuming. As a result, works that apply the optimization directly on the physical robot often evaluate few individuals along with a few generations, which reduces the competence of the evolutionary methods. For instance in \cite{faina2017automating}, controllers for a small tracked robot \textit{Pololu Zumo 32U4} \footnote{Pololu. Pololu Zumo 32u4 robot, 2016. \url{https://www.pololu.com/category/170/zumo-32u4-robot}} have been evolved with a population of 15 individuals during 20 generations and 2 simulations run, that is 600 evaluations during the entire optimization process. Adding to that the time of the actual evaluation of each controller resulted in 48 hours of running time.

For this reason, simulators in ER are often an appealing way to speed up the optimization process. Accurate simulation models are designed in order to evaluate the fitness in a fully virtual set-up. Accurate simulation models are designed in order to evaluate the fitness in a fully virtual set-up. Even though accurate models are build that corresponds to reality and successful controllers are evolved, the results are often sealed in the simulated world because of inaccuracies of the simulator or bad modeled of physical features in the simulator like friction, light, aerodynamics, etc. This transfer phenomenon is called \textit{reality gap}. Reality gap remains a critical issue that prevents the use of ER for practical applications in robotics. With my thesis I want to explore/compare state-of-the-art approaches to the \textit{reality gap} problem. In particular, the focus is on three main approaches:

\begin{enumerate}
    \item Reality-based optimization process where the entire optimization takes place fully on the physical robot.
    \item Simulation-based optimization process with the entire optimization in simulation.
    \item Robot in the loop optimization process, that optimize solutions in simulation but allows transfer experiments during the evolutionary process.
\end{enumerate}

Our first essential focus is on the robot in the loop optimization process, concretely the \textit{transferability approach} \cite{koos2012transferability} which is currently the best state-of-the-art approach to bypass the reality gap. However, instead of attempting to conduct as few transfers as possible, the \textit{transferability measure} is obtained for N solutions in every generation. We hypothesize that if the \textit{surrogate model} is more accurate and updated more frequently, this would allow boosting the evolutionary search. Another insight is concerned with an accurate simulation model, which is designed to accurately mimic the dynamics of the physical environment. Optimal solutions that emerge in simulation should transfer well onto the physical robot and achieve good performance/behavior in reality. In order to quantify how well a controller evolved in simulation transfer to reality, we define a \textit{transferable measure} which compares the corresponding real and simulated behavior.

Additionally, our Evolutionary strategy consists of two approaches. The first one is concerned with evolving weights of a feedforward neural network. In the second approach, instead of evolving just the weights of ANN controllers with a fixed structure we are also optimizing ANN controllers using NEAT \cite{stanley2002evolving}. A genetic algorithm for evolving both weights and topology of artificial neural networks.

All of the above approaches will be systematically compared and validated to on a robotic application, particularly obstacle avoidance task. As we need to evaluate a lot of controllers in the real world, which is time-consuming, we build an automated robotic platform which enables us to test and validate as many transfers as we want and run the entire optimization process on the physical robot without human intervention while taking the entire physical environment into account. Likewise, a simulation model will be designed from scratch that properly describes the dynamics of the real environment.

\section{Evolutionary Algorithms and Neuroevolution}

\section{The Reality Gap}

\section{The Transferability Approach}